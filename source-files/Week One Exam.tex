\documentclass[11pt]{exam}

% Format chemical formulae
\usepackage{mhchem}

\firstpageheader{}{\bf\Large Week One Test}{\Large Summer 2011}

\begin{document}
\begin{questions}

% Use Roman numerals in lists
\renewcommand{\theenumi}{\Roman{enumi}}
\renewcommand{\labelenumi}{\theenumi}

\question How many nanometers make up one meter?
\begin{choices}
    \choice $10^3$
    \choice $10^6$
    \choice $10^9$
    \choice $10^{12}$
\end{choices}
\answerline

\question How many significant figures are in the number 0.010010?
\begin{choices}
    \choice 2
    \choice 3
    \choice 4
    \choice 5
\end{choices}
\answerline

\question Which of the following are chemical (as opposed to physical) properties?

\hspace{.5cm}
\fbox{
\begin{minipage}{0.5\textwidth}
\begin{enumerate}
    \item The boiling point of ethyl alcohol is $78^{\circ}\mathrm{C}ø$.
    \item Oxygen combusts when heated.
    \item Diamond is very hard.
    \item Sugar ferments to form ethyl alcohol.
\end{enumerate}
\end{minipage}
}

\begin{choices}
    \choice I only.
    \choice I and III only.
    \choice II and IV only.
    \choice I, II, and IV only.
    \choice I, II, III, and IV.
\end{choices}
\answerline

\question What is the specific heat of water?
\begin{choices}
    \choice 4.184 $\frac{J}{g ^{\circ}\mathrm{C}}$
    \choice 7.531 $\frac{J}{g ^{\circ}\mathrm{C}}$
    \choice 2.680 $\frac{J}{g ^{\circ}\mathrm{C}}$
    \choice 9.373 $\frac{J}{g ^{\circ}\mathrm{C}}$
\end{choices}
\answerline

\pagebreak
\question Which of the following is the same in all isotopes of an element?

\begin{choices}
    \choice Number of protons
    \choice Number of neutrons
    \choice Number of electrons
    \choice Atomic mass
\end{choices}
\answerline

\question Which of the following has the largest atomic radius?

\hspace{.5cm}
\begin{tabular}{ | l | r | }
    \hline
    Li & Be \\ 
    3 & 4 \\ \hline
    Na & Mg \\ 
    11 & 12 \\ \hline
\end{tabular}

\begin{choices}
    \choice Li
    \choice Na
    \choice Be
    \choice Mg
\end{choices}
\answerline

\question Which is the correct order of these columns of the periodic table, from left to right?

\hspace{.5cm}
\fbox{
\begin{minipage}{0.5\textwidth}
\begin{enumerate}
    \item Alkaline earth metals
    \item Halogens
    \item Noble gases
    \item Transition metals
\end{enumerate}
\end{minipage}
}

\begin{choices}
    \choice IV, III, I, II.
    \choice I, IV III, II.
    \choice IV, I, II, III.
    \choice I, IV, II, III.
    \choice II, IV, III, I.
\end{choices}
\answerline

\question Which numbers correctly balance the equation below?

\hspace{.5cm}
\fbox{\ce{C2H7N + O2 -> CO2 + H2O + NO}}

\begin{choices}
    \choice 2, 4, 4, 7, 2
    \choice 2, 8, 4, 7, 2
    \choice 8, 34, 4, 28, 8
    \choice 4, 17, 8, 14, 4 
\end{choices}
\answerline

\end{questions}
\end{document}