\documentclass[11pt]{exam}
\usepackage{listings}
\usepackage{mhchem}
\usepackage{amsmath}

\firstpageheader{\bf\Large}{\bf\Large Chemistry Homework}{\Large Summer 2011}
\addpoints

\begin{document}

\vspace{0.1in} 
\hbox to \textwidth{Name:\enspace\hrulefill} 

% Questions start here:
\begin{questions}

\question Write the following numbers in scientific notation:
\vspace{5mm}

$212,000$
\vspace{5mm}

$0.0031300$
\vspace{5mm}

\question Write the following numbers in standard (non-scientific) notation:
\vspace{5mm}

$6.440 \times 10^3$
\vspace{5mm}

$9.58 \times 10^{-2}$
\vspace{5mm}

\question Find the number of significant figures in the answer to this calculation:
\vspace{5mm}

$2000.0 - 1999.0$
\vspace{5mm}

\question Name what the following units measure:
\vspace{5mm}

Kilograms
\vspace{5mm}

Liters
\vspace{5mm}

Meters
\vspace{5mm}

$\dfrac{g}{mL}$
\vspace{5mm}

\question Write whether the following properties are chemical or physical:
\vspace{5mm}

The malleability of aluminum foil
\vspace{5mm}

The melting point of ice cream
\vspace{5mm}

The combustion of paper
\vspace{5mm}

\pagebreak
\question Write whether the following are homogenous mixtures, heterogeneous mixtures, or pure substances:
\vspace{5mm}

Gasoline
\vspace{5mm}

Copper
\vspace{5mm}

Air
\vspace{5mm}

Salt water
\vspace{5mm}

A stream with gravel at the bottom
\vspace{5mm}

\question Using the formula for specific heat, calculate the amount of energy needed to raise the temperature of 1.6g of gold (specific heat $0.13 \frac{J}{g^{\circ}C}$) from $23 ^{\circ}C$ to $41 ^{\circ}C$:
\vspace{5mm}

\end{questions}
\end{document}